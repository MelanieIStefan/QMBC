\documentclass[pdflatex]{article}

\begin{document}

\section*{Choose a number out of 4}

\begin{enumerate}
\item
Have N people pick a number from 1 to R, inclusive. What is the probability, p, of K or more choices of any single number? Your function should take inputs R, N, K and a 4th variable, nSims (number of simulations to run) and return the probability, p.

\verb_p = MyFunction(R, N, K, nSims);_
\item
This example is from actual class data taken on 16 August 2012. The instructions were to pick at random a number from \(1\) to \(4\). Here is the actual data from \(75\) respondents:

\begin{itemize}
\item
7 people chose number 1 (\(9\,\%\))
\item
24 people chose number 2 (\(32\,\%\))
\item
34 people chose number 3 (\(45\,\%\))
\item
10 people chose number 4 (\(13\,\%\))
\end{itemize}

Is this distribution random?

\emph{Hint}: The hardest part of this exercise is defining what class of outcomes you would consider "weird." You need to have a precise definition so that you can count how often they occur when you generate a bunch of simulations under the null hypothesis (H0). By the way, what is your null hypothesis? Start out by discussing - in words - in small groups the definition of H0 and the precise definition of the type of event you would consider to violate H0. \emph{Resist the temptation to start coding right away.}

\item
Bonus exercise: For MATLAB show-offs, this can be done in a single line of code. What are the advantages and disadvantages of doing it this way?

\end{enumerate}

\end{document}
